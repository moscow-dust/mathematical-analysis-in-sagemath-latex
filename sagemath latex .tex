documentclass{article}
% set font encoding for PDFLaTeX, XeLaTeX, or LuaTeX
\usepackage{ifxetex,ifluatex}
\usepackage[utf8]{inputenc}
\usepackage[english, russian]{babel}
\usepackage[T1, T2A]{fontenc}
\if\ifxetex T\else\ifluatex T\else F\fi\fi T%
  \usepackage{fontspec}
\else
  \usepackage[T1]{fontenc}
  \usepackage[utf8]{inputenc}
  \usepackage{lmodern}
\fi

\usepackage{hyperref}

\title{Курсовая}
\author{Чернова Анастасия}

% Enable SageTeX to run SageMath code right inside this LaTeX file.
% http://doc.sagemath.org/html/en/tutorial/sagetex.html
% \usepackage{sagetex}

% Enable PythonTeX to run Python – https://ctan.org/pkg/pythontex
\usepackage{sagetex}

\begin{document}
\maketitle
\newpage
\section{задача, вариант27}
\begin {sagesilent}
x = var('x')
y(x) = (arctan(sqrt(x)))^x
plot.options["ymin"]=0
plot.options["ymax"]=15
p = plot(y,(0,15))
chet = y(-x).simplify()!=y(x).simplify()
no_chet = y(-x).simplify()!=-y(x).simplify()
chet_b = bool(y(-x).simplify()!=y(x).simplify())
no_chet_b = bool(y(-x).simplify()!=-y(x).simplify())
T = var('T')
period = (arctan(sqrt(x-T)))^(x-T)!= (arctan(sqrt(x+T)))^(x+T)
p1 = plot(y,(-15,15))+ point((0,1),size = 40, color = 'green')
dy = y(x).derivative()
dy2=dy(x).derivative()
root = find_root(dy,0,2)
p_dy = plot(dy,(-15,15))
p_dy2 = plot(dy2,(-15,15))
l_i = limit(y(x), x=oo)
l_z = limit(y(x), x=0)
l_ass = limit(y(x)/x, x = oo)
p3 = plot(0, xmin = 0, xmax = 2, axes = False,aspect_ratio = 0.05)+point((0,0), color = 'green', size = 40, zorder = 10)+text(r"+", (1, 1), color = 'black', fontsize = 20)+text("0",(0,2), color = 'red', fontsize = 13)
p4 = plot(0, xmin = 0, xmax = 2, axes = False, aspect_ratio = 0.05)+point((0,0), color = 'green', size = 40, zorder = 10)+text(chr(92), (0.35, 1), color = 'black',fontsize = 20)+point((root, 0), color = 'green', size = 40, zorder = 10)+text("/", (1.2, 1), color = 'black',fontsize = 20)+text("/", (1.2, 1), color = 'black',fontsize = 20)+text(str(numerical_approx(root, digits =2)),(0.7,1), color = 'red', fontsize = 13)+text("0",(0,1), color = 'red', fontsize = 13)
l_b=bool(limit(y, x =+0)==limit(y, x = -0))
\end{sagesilent}
\large $\sage{(y(x))}$

график от 0 до 15

\begin{center}
\sageplot[scale=0.8]{p}
\end{center}
\large $\sage{(chet)}$

\large $\sage{(no_chet)}$

\large $\sage{(chet_b)}$

не является четной

\large $\sage{(no_chet_b)}$


не является нечетной

\large $\sage{(period)}$

не является периодической

точка пересечения с осью Y
x =
\large $\sage{(y(0))}$
\begin{center}
\sageplot[scale=0.8]{p1}
\end{center}
первая производная
\large $\sage{(dy(x))}$
вторая производная\\
\scalebox{.7}{$\sage{(dy2(x))}$}
точка экстремума и значение функции в этой точке
\large $\sage{(round(root,4))}$ $\sage{(round(y(root),4))}$
\begin{center}
\sageplot[scale=0.7]{p_dy}
\end{center}
\begin{center}
\sageplot[scale=0.7]{p_dy2}
\end{center}
$\lim_{x\to\infty} y(x)=$ $\sage{(l_i)}$
$\lim_{x\to 0} y(x)=$ $\sage{(l_z)}$
$\lim_{x\to\infty} y(x)/x=$ $\sage{(l_ass)}$\\
нет ассимптот\\
промежутки знакопостоянства
\begin{center}
\sageplot[scale=0.7]{p3}
\end{center}
промежутки возрастания и убывания
\begin{center}
\sageplot[scale=0.7]{p4}
\end{center}
$\lim_{x\to +0} y(x)$=$\lim_{x\to-0} y(x)=$1
\large $\sage{(l_b)}$
точек разрыва нет
\section{задача}
\begin {sagesilent}
matr = matrix(QQ,3,3,[1,2,3,2,4,6,3,1,-1]);matr
b = vector([4,3,1])
delta_1=copy(matr)
delta_2=copy(matr)
delta_3=copy(matr)
delta_1[:,0] = b
delta_2[:,1] = b
delta_3[:,2] = b
d = matr.det()
expand_matr=matrix(QQ,3,4,[1,2,3,4,2,4,6,3,3,1,-1,1]);matr
step = expand_matr.echelon_form()
ex_r = expand_matr.rank()
m_r = matr.rank()
\end {sagesilent}
Начальная матрица A =$\sage{(matr)}$\\
столбец свободных членов b =$\sage{(b)}$\\
ищем дельты\\
$\Delta$1 = $\sage{(delta_1)}$
$\Delta$2 = $\sage{(delta_2)}$
$\Delta$3 = $\sage{(delta_3)}$\\
определитель начальной матрицы\\
$\Delta$ = $\sage{(d)}$
невозмжно решить методом Крамера\\
Метод гаусса\\
расширенная матрица
$\sage{(expand_matr)}$
ступенчатая форма
$\sage{(step)}$
x1-x3=0
x2+x3=0
0=1
решений нет\\
Ранг матрицы системы = $\sage{(m_r)}$ ранг матрицы системы=$\sage{(ex_r)}$\\
Они не равны, значит, по теореме Кронеккера-Капелли система несовместна
\section{задача}
\begin {sagesilent}
A_matr= matrix(QQ,3,3,[-2,4,-6,-1,0,-2,4,4,2]);A_matr
B_matr=matrix(3,3,[2,8,-9,-1,1,4,1,0,2]);B_matr

\end{sagesilent}
исходные матрицы матричного уравнения\\
A =$\sage{(A_matr)}$\\
B =$\sage{(B_matr)}$\\
матричное уравнение\\
$\frac{1}{2}$X$\mathbf{(A)}^T$=$2B^2$\\
\begin {sagesilent}
A_matr=(1/2)*A_matr.transpose()
B_matr= 2*B_matr^2
x=B_matr*A_matr.inverse()
\end{sagesilent}
приведем к виду XA=B\\
X$\sage{(A_matr)}$=$\sage{(B_matr)}$\\
$X = B{A}^{-1}$\\
Решение матричного уравнения\\
X =$\sage{(x)}$
\section{задача вариант 27}
\begin {sagesilent}
x = var('x')
eq = 7*x^3-4*x^2+5*x-3==0
poly = 7*x^3-4*x^2+5*x-3
y = var('y')
eq_ins=(eq.subs(x==y+4/(3*7)).expand())
p = 89/147
q = -2837/9261
Q=(p/3)^3+(q/2)^2
a=((-q/2+Q^(1/2))^(1/3)).n(digits=3)
b = -(abs(-q/2-sqrt(Q))^(1/3)).n(digits = 3)
y1 = a+b+(4/21)
y2 = -(a+b)/2+4/(3*7)-i*(a-b)/2*1.732
y3 =-(a+b)/2+4/(3*7)+i*(a-b)/2*1.732
p5 = plot(7*x^3-4*x^2+5*x-3, ymin = -15)+point((y1,0), size = 22, color = 'red')
phi_1 = atan2(y2.imag(), y2.real())
r_1 = y2.abs()
y_trig_1 = r_1*(cos(phi_1, hold=True) + I*sin(phi_1, hold=True))
phi_2 = atan2(y3.imag(), y3.real())
r_2 = y3.abs()
y_trig_2 = r_2*(cos(phi_2, hold=True) + I*sin(phi_2, hold=True))
y_exp_1 = r_1*exp(i*phi_1, hold = True)
y_exp_2 = r_2*exp(i*phi_2, hold = True)
\end{sagesilent}
\large $\sage{(eq)}$\\
пoдстановка \large x = y - $\frac{b}{3a}$\\
\large $\sage{(eq_ins)}$\\
p = $\frac{89}{147}$
q = $\frac{-2837}{9261}$\\
\large Q = $(\frac{p}{3})^3$+$(\frac{q}{2})^2$\\
Q = \large $\sage{(Q)}$\\
a = \large $\sage{(a)}$\\
b = \large $\sage{(b)}$\\
y1 = \large $\sage{(y1)}$\\
y2 = \large $\sage{(y2)}$\\
y3 = \large $\sage{(y3)}$
\begin{center}
\sageplot[scale=0.7]{p5}
\end{center}
тригнометрическая форма y2: \large $\sage{(y_trig_1)}$\\
тригонометрическая форма y3: \large $\sage{(y_trig_2)}$\\
экспоненциальная форма y2: \large $\sage{(y_exp_1)}$\\
экспоненциальная форма y3: \large $\sage{(y_exp_2)}$\\
\section{задача вариант 27}
\begin {sagesilent}
x = var('x')
eq = x^4 - 2*x^3 + 3*x^2 - 7*x + 1
var("y")
eq_rep = eq.subs(x==y+1/2).expand()
pqr = {'p':3/2, 'q': -5, 'r': -31/16}
var("s p q r")
eq_s = 2*s^3 - p*s^2 - 2*r*s + r*p - q^2/4
eq_s_n = eq_s(**pqr)
sols = solve(eq_s_n, s)
s_0 = sols[2].rhs()
var("y s p q")
poly_y_1 = y**2 - y*sqrt(2*s - p) + q/(2*sqrt(2*s - p)) + s
poly_y_2 = y**2 + y*sqrt(2*s - p) - q/(2*sqrt(2*s - p)) + s
poly_y_1_n = poly_y_1(**pqr, s=s_0)
poly_y_2_n = poly_y_2(**pqr, s=s_0)
sols = solve(poly_y_1_n, y)
sols.extend(solve(poly_y_2_n, y))
phi_1 = atan2((sols[2].rhs().n(digits=5)+0.5).imag(), (sols[2].rhs().n(digits=5)+0.5).real())
r_1 = (sols[2].rhs().n(digits=5)+0.5).abs()
x_trig_1 = r_1*(cos(phi_1, hold=True) + I*sin(phi_1, hold=True))
phi_2 = atan2((sols[3].rhs().n(digits=5)+0.5).imag(), (sols[3].rhs().n(digits=5)+0.5).real())
r_2 = (sols[3].rhs().n(digits=5)+0.5).abs()
x_trig_2 = r_2*(cos(phi_2, hold=True) + I*sin(phi_2, hold=True))
x_exp_1 = r_1*exp(i*phi_1, hold = True)
x_exp_2 = r_2*exp(i*phi_2, hold = True)
p14 = plot(x^4 - 2*x^3 + 3*x^2 - 7*x + 1, ymin = -5, xmax = 5)+point(((sols[0].rhs().n(digits=5)+0.5),0), size = 22, color = 'red')+point(((sols[1].rhs().n(digits=5)+0.5),0), size = 22, color = 'red')
\end {sagesilent}
$\sage{(eq)}$
замена
x = y-$\frac{a}{4}$\\
$\sage{(eq_rep)}$\\
p = $\frac{3}{2}$ q = -5 r = -$\frac{31}{16}$\\
резольвента $\sage{(eq_s_n)}$\\
найдем ненулевое решение и получим квадратные уравнения:\\
$\sage{(poly_y_1)}$\\
$\sage{(poly_y_2)}$\\
Подставим в них найденное решение и решим эти уравнения:\\
$x_1$ = $\sage{(sols[0].rhs().n(digits=5) +0.5)}$ $x_2$ = $\sage{(sols[1].rhs().n(digits=5) +0.5)}$ $x_3$ = $\sage{(sols[2].rhs().n(digits=5) +0.5)}$ $x_4$ = $\sage{(sols[3].rhs().n(digits=5) +0.5)}$\\
корни в тригонометрической и экспоненциальных формах\\
$\sage{(x_trig_1)}$\\
$\sage{(x_trig_2)}$\\
$\sage{(x_exp_1)}$\\
$\sage{(x_exp_2)}$\\
график с корнями
\begin{center}
\sageplot[scale=0.7]{p14}
\end{center}
\section{задача}
\begin {sagesilent}
f(x) = -6*x^5+41*x^4-53*x^3+15*x^2-7*x+4
g(x) = 14*x^5-13*x^4+10*x^3+4*x^2-24*x+9
r = f.maxima_methods().divide(g)
tmp_1 = r[0]
if r[1]!=0:
    r = g.maxima_methods().divide(r[1])
    tmp_2=r[0]
else:
    tmp = f(x)
while r[1]!=0:
    tmp = r[1]
    r=tmp.maxima_methods().divide(r[1])
tmp_raz = f*tmp_2+g*(1+tmp_1*tmp_2)
gcd_res = gcd(f,g)
\end{sagesilent}
исходные многочлены\\
f(x) = \large $\sage{(f(x))}$\\
g(x) = \large $\sage{(g(x))}$\\
НОД = \large $\sage{(tmp)}$\\
НОД через gcd()
\large $\sage{(gcd_res(x))}$\\
разложение по тождеству Безу:\\
\large $\sage{(tmp)}$ = \large $\sage{(tmp_raz(x))}$\\
\section{задача вариант 27}
\begin {sagesilent}
A=matrix(3,3,[5,4,-2,-2,2,5,1,3,4])
E = matrix(3,3,[1,1,0,0,1,1,1,0,1])
A1 = E.inverse()*A*E

char_A =A.charpoly()
char_A1=A1.charpoly()
char_r = x^3-11*x^2+33*x-33==0
y = var('y')
eq_ins=(char_r.subs(x==y+11/3)).expand()
show(eq_ins)
p = -22/3
q = -286/27
Q=(p/3)^3+(q/2)^2
show(Q)
a=((-q/2+Q^(1/2))^(1/3)).n(digits=4)
b = (abs(-q/2-sqrt(Q))^(1/3)).n(digits = 4)
y1 = a+b+11/3
y2 = -(a+b)/2+11/3-i*(a-b)/2*1.732
y3 =-(a+b)/2+11/3+i*(a-b)/2*1.732
A_y1=matrix(3,3,[5/2-y1,13/2,1,1/2,5/2-y1,1,5/2,-5/2,6-y1])
ech_1= A_y1.echelon_form()
A_y2=matrix(3,3,[(5/2-y2),13/2,1,1/2,(5/2-y2),1,5/2,-5/2,(6-y2)])
ech_2= A_y2.echelon_form()
A_y3=matrix(3,3,[(5/2-y3),13/2,1,1/2,(5/2-y3),1,5/2,-5/2,(6-y3)])
ech_3= A_y3.echelon_form()
\end {sagesilent}
исходная матрица преобразования
A = \large $\sage{(A)}$\\
базис
E = \large $\sage{(E)}$\\
формула перехода к другому базису\\
$A1={E}^{-1}AE$\\
матрица преобрзования в новом базисе\\
A1 = \large $\sage{(A1)}$\\
полином А: \large $\sage{(char_A)}$\\
полином А1: \large $\sage{(char_A1)}$\\
характеристические многочлены равны\\
корни характеристического многочлена:\\
y1 = \large $\sage{(y1)}$
y2 = \large $\sage{(y2)}$
y3 = \large $\sage{(y3)}$\\
матрицы с корнями\\
\large $\sage{(A_y1)}$\\
\large $\sage{(A_y2)}$\\
\large $\sage{(A_y3)}$\\
собственный вектор (0,0,0)
\section{задача}
\begin {sagesilent}
y = var('y')
z=var('z')
x = var('x')
f(x,y,z) = 8*x^2-4*y^2+3*z^2-2*x*y+2*x*z-2*y*z+7*x+8*y+9*z-10
p6 = implicit_plot3d(f,(x,-5,5), (y,-5,5), (z,-5,5))
A = matrix(3,3,[8,-1,1,-1,-4,-1,1,-1,3]);A
eig = A.eigenvalues()
t1 = 7
t2 = -23
beta=A.det()
B = matrix(4,4,[8,-1,1,7/2,-1,-4,-1,4,1,-1,3,9/2,7/2,4,9/2,-10]);B
d = B.det()
a = -d/((eig[1].n(digits = 4))*beta)
b = -d/((eig[2].n(digits = 4))*beta)
c = d/((eig[0].n(digits = 4))*beta)
eq = (x^2)/a+(y^2)/b-(z^2)/c==1
p16 = implicit_plot3d(eq,(x,-5,5), (y,-5,5), (z,-5,5))
\end{sagesilent}
\large $\sage{(f(x,y,z))}$
\begin{center}
\sageplot[scale=0.6]{p6}
\end{center}
составим матрицу коэффициентов:
\large $\sage{(A)}$\\
собственные значения
$\sage{(eig[0].n(digits = 4))}$ $\sage{(eig[1].n(digits = 4))}$ $\sage{(eig[2].n(digits = 4))}$\\
$t_1 = 7  t_2 = -23 beta = -101$\\
расширенная матрица коэффициентов
$\sage{(B)}$\\
d = $\sage{(d)}$\\
$beta != 0$ $t_2<0$ $d>0$ значит это однополостный гиперболоид\\
коэффициенты в каноническом виде\\
$a^2$ = $\sage{(a)}$
$b^2$ = $\sage{(b)}$
$c^2$ = $\sage{(c)}$\\
уравнение в каноническом виде\\
$\sage{(eq)}$
\begin{center}
\sageplot[scale=0.6]{p16}
\end{center}
\section{задача вариант 27}
\begin {sagesilent}
f(x)= x^2*ln(x)
p8=plot(x^2*ln(x), x, (0, 3),ymax=3, color = 'green')+plot(x^2*ln(x), x, (1, 2),ymax=3,fill = True, fillcolor = 'pink', color='green')+parametric_plot((1,x),(x,0,3), linestyle = '--', color = 'red', ymax = 3)+parametric_plot((2,x),(x,0,3), linestyle = '--', color = 'red', ymax =3)
num =numerical_integral(x^2*ln(x),1, 2)
def trap(a,b,n, f,p):
    width= (b-a)/n
    integral = 0
    i =0
    tmp=p
    while i < n:
        x1 = a + i*width
        x2 = a + (i+1)*width
        pol = polygon([[x1,0],[x1, f(x1)], [x2, f(x2)],[x2,0]], color = 'orange',thickness=4,fill=False)
        i=i+1
        integral += 1/2*(x2-x1)*(f(x1) + f(x2))
        t = text(r"i={}, curr_square={}, result={}".format(str(i), str(1/2*(x2-x1)*(f(x1) + f(x2)).n(digits=3)),str(numerical_approx(integral).n(digits =3))),(-2,0.5), fontsize=11, color="black")
        (p+pol+tmp+t).show()
        p9 = p+pol+tmp+t
        tmp = tmp+pol
    return [integral, p9]
integ_1,p9 = trap(1,2,6 ,f(x),p8)

def rect(a,b,n,f, p):
    h = (b-a)/n
    integral = 0
    x2 =a
    i=0
    tmp = p
    while(i<n):
        x1 = x2 + h/2;#я беру середину прямоугольника, говорят, так быстрее считается
        x2 = x2 + h;
        integral += f(x1)*h;
        pol = polygon([[x1,0],[x1, f(x1)], [x2, f(x1)],[x2,0]], color = 'orange',thickness=4,fill=False)
        i=i+1
        t = text(r"i={}, curr_square={}, result={}".format(str(i), str((f(x1)*h).n(digits=3)),str(numerical_approx(integral).n(digits =3))),(-2,0.5), fontsize=11, color="black")
        show(p+pol+tmp+t)
        p10 = p+pol+tmp+t
        tmp = tmp+pol
    return [integral, p10]
integ_2, p10 = rect(1,2,6,f(x),p8)
\end {sagesilent}
Функция: 
\large $\sage{(f(x))}$
Площадь под графиком:\\
\begin{center}
\sageplot[scale=0.9]{p8}
\end{center}
возмем просто интеграл:\\
$\int_{1}^{2}$ $\sage{(f(x))}$
dx = $\sage{(num[0])}$\\
\begin{center}
\sageplot[scale=0.9]{p9}
\end{center}
интеграл методом трапеций:
$\sage{(integ_1.n(digits = 5))}$
\begin{center}
\sageplot[scale=0.9]{p10}
\end{center}
интеграл методом прямоугольников:
$\sage{(integ_2.n(digits = 5))}$
\section{задача вариант 27}
\begin {sagesilent}
x = var('x')
f(x)=(3*sin(x))^2+2*(2*sin(2*x))^3-9*x
p11 =plot(f(x), x, (-10,10),ymax = 20,ymin = -20,color = 'green' )
def nu(a, b, eps, f, p):
    dx=f.derivative()
    i=0
    x = var('x')
    d2x = dx.derivative()
    if f.subs(x==a)*d2x.subs(x==a)>0:
        x0 = a
    else:
        x0=b
    while abs(f.subs(x==x0)/dx.subs(x==x0))>eps:
        i+=0.1
        p += plot(dx(x0)*(x-x0)+f(x0), x, (-10,10),color = (0.7,0.5+i,0.1+i))
        x0 = x0-f.subs(x==x0)/dx.subs(x==x0)
    p += plot(dx(x0)*(x-x0)+f(x0), x, (-10,10),color = (0.7+i,0.5,0.1+i))
    p=p+point((x0,0), size =38, color = 'red', ymin = -20, xmin = -5)
    return [x0, p]
root, p12 = nu(1,2,0.1, f(x),p11)
dx=f(x).derivative()
sign = bool(f(2)*f(1)<0)
\end {sagesilent}
$\sage{(f(x))}$
\begin{center}
\sageplot[scale=0.9]{p11}
\end{center}
ищем корень на отрезке $[1,2]$ точность $0.1$\\
x = $\sage{(root.n(digits = 4))}$
\begin{center}
\sageplot[scale=0.9]{p12}
\end{center}
проверим применимость метода на отрезке:\\
производная функции $\sage{(dx)}$\\
$show(find\_root(dx,1,2))$ здесь будет ошибка, потому что корня нет и значит функция монотонна на отрезке\\
функция непрерывна и определена на отрезке[1;2]поскольку синусы и линейная функция непрерывны на области определения\\
функция на кнцах отрезка имеет разные знаки\\
$\sage {(f(1)*f(2))}<0$
$\sage {(sign)}$
\end{document}
